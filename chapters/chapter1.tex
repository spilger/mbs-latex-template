% !TEX root =  master.tex
\chapter{Main EU taxonomy requirements for NFEs}
Article 3 of the EU Taxonomy sets out four criteria for environmentally sustainable economic activities.

The first criterion in Article 3(a) states that an economic activity must make a significant contribution to one or more of the environmental objectives set out in Articles 10 to 16. These objectives include climate change mitigation and adaptation, the sustainable use and protection of water and marine resources, the transition to a circular economy, pollution prevention and control, and the protection and restoration of biodiversity and ecosystems. This contribution is assessed on the basis of specific indicators and thresholds set out in the technical screening criteria to ensure that the activity is in line with the overall objectives of sustainability and environmental protection.
\autocites[Cf.][Articles 3, 10-16]{eu_regulation}

The second criterion is that the activity must not significantly harm another environmental objective referred to in Article 9, which is compatible with Article 17. This principle is known as \ac{DNSH}. According to Article 17, an activity is considered to significantly harm environmental objectives if it leads to significant adverse effects on any of the following objectives: Climate change mitigation, adaptation to climate change, sustainable use and protection of water and marine resources, transition to a circular economy, pollution prevention and control, and protection and restoration of biodiversity and ecosystems.
\autocites[Cf.][Articles 3, 9, 17]{eu_regulation}


The third criterion is that the activity is carried out in compliance with the minimum guarantees defined in Article 18. These minimum safeguards ensure that the activity complies with high social and corporate governance standards and, in particular, with guidelines such as the OECD Guidelines for Multinational Enterprises and the UN Guiding Principles on Business and Human Rights. This compliance is crucial in order to maintain the integrity and ethical standards of the activity and at the same time contribute to environmental sustainability.\autocites[Cf.][Article 3, 18]{eu_regulation}


The fourth criterion states that a regulation must fulfil the technical screening criteria of Articles 10(3), 11(3), 12(2), 13(2), 14(2) or 15(2). These technical screening criteria contain detailed benchmarks and thresholds that an activity must fulfil in order to be considered as contributing substantially to one or more of the environmental objectives, such as climate change mitigation, climate change adaptation, sustainable use and protection of water and marine resources, transition to a circular economy, pollution prevention and control, and protection and restoration of biodiversity and ecosystems. In particular the following\autocites[Cf.][Article 3, 12-15]{eu_regulation}:
\begin{itemize}
    \item \textbf{Article 10 (3)}: Climate change mitigation, such as improving energy efficiency or producing renewable energy.
    \item \textbf{Article 11 (3)}: Climate change adaptation, including measures that reduce adverse climate impacts.
    \item \textbf{Article 12 (2)}: Sustainable use and protection of water and marine resources, enhancing water quality.
    \item \textbf{Article 13 (2)}: Transition to a circular economy, emphasizing resource efficiency and waste reduction.
    \item \textbf{Article 14 (2)}: Pollution prevention and control, minimizing emissions and waste.
    \item \textbf{Article 15 (2)}: Protection and restoration of biodiversity and ecosystems, supporting natural habitats.
\end{itemize}


According to Article 3 of the EU Taxonomy, the assessment of the environmental sustainability of an economic activity includes three steps: Firstly, specific indicators and thresholds set out in the technical screening criteria are used to assess whether the activity contributes significantly to one or more environmental objectives. Secondly, similar indicators and thresholds are used to ensure that the activity does not significantly harm any of the other environmental objectives in accordance with the \ac{DNSH} principle. Thirdly, confirmation that the activity complies with the minimum protection requirements, whereby the fulfilment of all these criteria results in the activity being classified as "taxonomy-compliant". 
\autocites[Cf.]{eu_regulation}[Cf.][10]{tax1}

