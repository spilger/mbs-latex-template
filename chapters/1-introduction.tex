% !TEX root =  master.tex
\chapter{Introduction}
The European Taxonomy, published in 2020, aims to establish a system of sustainability targets for the European Union. It does this by applying technical criteria to verify whether economic activities can be considered sustainable. This taxonomy is important to provide investors with standardised guidelines that enable comparability and to promote sustainable investments. It also serves to check whether companies are organising their activities in line with the European Green Deal. Another aim is to reduce the risk of greenwashing by establishing clear criteria. This taxonomy also helps companies to plan and implement their transition processes towards greater sustainability. It is on question to what extent the EU taxonomy contributes to it's objectives. This happens due to several reasons. For certain economic sectors the criteria is really clear, while it is not clear for others. Fore example the automotive sectors has really clear criteria while the also emission intensive materials sector may not have sufficient criteria.
\autocites[Cf.][1-4]{tax1}[Cf.]{eu_tax_sus}

These questions are still unclear, so this work will explain the main criteria of Regulation 2020/852 of the European Parliament and of the Council of 18 June 2020, hereafter referred to as the EU taxonomy. The example of \ac{SAP} is then analysed to see how the company presents the taxonomy in its report, how much of its business is covered by the taxonomy and how it reports taxonomy-aligned revenue. This information is compared with other figures to find out how suitable the EU taxonomy is for assessing the sustainability of \ac{SAP}.