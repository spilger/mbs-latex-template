% !TEX root =  master.tex
\chapter{SAPs presentation of the reporting requirements}
This chapter provides an analysis of how SAP, a leading multinational software corporation, aligns its reporting with the EU Taxonomy Regulation. It will be explored the extent of SAP's business coverage under the EU Taxonomy, the reporting of taxonomy-aligned revenue and capital expenditures (CAPEX), and evaluated the suitability of the taxonomy as a tool for assessing the company's sustainability efforts.
\section{Coverage of business by EU taxonomy}
SAP identifies two taxonomy eligable activities in their report. The activities are 
\begin{itemize}
    \item Activity 8.1 “Data processing, hosting, and related activities” (substantial contribution to climate change mitigation) - SAP has assessed its data centres and cloud solutions, but has concluded that they do not meet the criteria for an adjustment to the taxonomy in 2023. The problems include the refrigerants, which have too high a potential for global warming, but whose immediate replacement would generate toxic waste, so it is not helpful to replace them immediately. In addition, the data centres, colocations and hyperscalers do not meet all the criteria. SAP is committed to achieving harmonisation within the next few years.
    \item Activity 3.3 “Manufacture of low carbon technologies for transport” (substantial contribution to climate change mitigation) - Here the SAP car fleet is controbuting to climate change mitigation. In 2023 SAP capitalized €210 million in this activity and reports €51 million as taxonomy aligned. SAP is commited to buy more taxonomy aligned vehicles in the future.
    \item Other Activities - Combined Heat/Cool and Power Generation Facilities which process fossil gaeous fuels are operated by SAP in single locations. SAP reports that the amount is negligible.
\end{itemize}
In numbers, this means that SAP has 43\% of taxonomy-eligible revenues, 19\% of taxonomy-eligible \ac{OPEX} and 27\% of taxonomy-eligible activities in \ac{CAPEX}.  In each category in which the business is measured according to the EU taxonomy, SAP is only represented with less than half to less than a quarter of the total business.
\autocites[Cf.][92-98]{sapreport}


\section{Reporting of taxonomy aligned revenue and CAPEX}
SAP reports taxonomy-aligned revenue and \ac{CAPEX} separately. SAP has 43\% taxonomy-eligible activities for \textbf{revenue}. None of them are taxonomy-aligned. This makes €13.361 million taxonomy-eligible but unaligned revenue. 57\% (€17.846 million) are not eligible for the EU taxonomy. The situation is different with the \textbf{\ac{CAPEX}}. Here, 27\% (€370 million) is taxonomy-eligible and 73\% (€1,008 million) is not taxonomy-eligible. 4\% of \ac{CAPEX} are taxonomy-eligible and aligned. This results from the above-mentioned activity 3.3, which relates to the electric cars in SAP's fleet.
\autocites[Cf.][92-98]{sapreport}

\section{Suitability of the taxonomy to judge the companies sustainability}
The analysis of the business areas covered by SAP under the EU Taxonomy shows that while the company has identified and reported on taxonomy-eligible activities, progress is still needed to bring these activities into taxonomy-compliant status. In the areas of data processing and fleet management in particular, considerable efforts are being made to fulfil the technical screening criteria and align with the environmental objectives set out in the EU taxonomy. If this is done based on the EU taxonomy, that would be good, but SAP already reports to be carbon neutral. In terms of resource consumption, waste and $CO_2e$ emissions, SAP has not only significantly reduced its environmental impact, but also claims to be climate neutral. This shows that the taxonomy is not suitable for assessing SAP's sustainability as a complete company, as SAP's business activities are only covered to a small extent, but it also shows that there are many technical details to think about when thinking about sustainability. SAP may not have focussed sufficiently on technical criteria, but merely sought to report carbon neutrality, even if that means not replacing refrigerants with those that produce less or no toxic waste. Therefore, technical criteria are needed to assess SAP's sustainability and it is good if business activities are included, but the percentage of business units included should be higher to make companies as a whole comparable and assessable.
\autocites[Cf.][110-115]{sapreport}[Cf.][276-278]{sapreport}