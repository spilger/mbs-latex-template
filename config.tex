% !TEX root = master.tex

%%%%%%%%%%%%%%%%%%%%%%%%%%%%%%%%%%%%%%%%%%%%%%%%%%%%%%%%%%%%%%%%%%
% ANLEITUNG:
% Passen Sie gegebenenfalls alle Stellen im Dokument an, die mit
% @stud
% markiert sind.
%%%%%%%%%%%%%%%%%%%%%%%%%%%%%%%%%%%%%%%%%%%%%%%%%%%%%%%%%%%%%%%%%%

% LANGUAGE SETTINGS
\usepackage[english]{babel}
\usepackage{csquotes}

% ESSENTIAL PACKAGES
\usepackage{makeidx}                  % allows index generation
\usepackage{listings}                 % Format Listings properly
\usepackage{lipsum}                   % Blindtext
\usepackage{graphicx}                 % use various graphics formats
\usepackage{varioref}                 % nicer references \vref
\usepackage{caption}                  % better Captions
\usepackage{booktabs}                 % nicer Tabs
\usepackage[hidelinks=true]{hyperref} % no red markings on links
\usepackage{fnpct}                    % Correct superscripts
\usepackage{calc}                     % Used for extra space below footsepline, in particular
\usepackage{array}
\usepackage{acronym}
\usepackage{algorithm}
\usepackage{algpseudocode}
\usepackage{setspace}
\usepackage{tocloft}
\usepackage[english]{babel}
\usepackage{csquotes}

% FONT SELECTION
\usepackage{fontspec}
\setmainfont{Times New Roman}
\doublespacing

% Adjust default family to serif (Times New Roman)
\renewcommand*{\familydefault}{\rmdefault}
\addtokomafont{disposition}{\rmfamily}

\usepackage{pgfplots}
\pgfplotsset{compat=newest}

\usepackage[a4paper, margin=1in]{geometry}
\pgfplotsset{width=14cm,height=10cm,compat=1.9}
\usepackage{array}
\newcolumntype{L}[1]{>{\raggedright\arraybackslash}p{#1}}
\newcolumntype{C}[1]{>{\centering\arraybackslash}p{#1}}


\usepackage{tikz}
\usetikzlibrary{positioning, arrows.meta}
%		BIBLIOGRAPHY SETTINGS
%

% Import in master.tex
%\AdaptNoteOpt\footcite\multfootcite   %will add  separators if footcite is called multiple consecutive times
%\AdaptNoteOpt\autocite\multautocite % will add  separators if autocite is called multiple consecutive times
% Uncomment the next line for IEEE-style
% \usepackage[backend=biber, autocite=inline, style=ieee]{biblatex} 	% Use IEEE-Style (e.g. [1])

% Uncomment the next line for alphabetic style
% \usepackage[backend=biber, autocite=inline, style=alphabetic]{biblatex} 	% Use alphabetic style (e.g. [TGK12])

% Uncomment the next two lines vor Harvard-Style
%\usepackage[backend=biber, style=apa]{biblatex}
%\DeclareLanguageMapping{german}{german-apa}


%\DefineBibliographyStrings{ngerman}{  %Change u.a. to et al. (german only!)
%	andothers = {{et\,al\adddot}},
%}

%%% Uncomment the following lines to support hard URL breaks in bibliography
%\apptocmd{\UrlBreaks}{\do\f\do\m}{}{}
%\setcounter{biburllcpenalty}{9000}% Kleinbuchstaben
%\setcounter{biburlucpenalty}{9000}% Großbuchstaben


\setlength{\bibparsep}{\parskip}		%add some space between biblatex entries in the bibliography
\addbibresource{bibliography.bib}	%Add file bibliography.bib as biblatex resource

%		FOOTNOTES
%
% Count footnotes over chapters
\usepackage{chngcntr}
\counterwithout{footnote}{chapter}

%	ACRONYMS
%%%
%%% WICHTIG: Installieren Sie das neueste Acronyms-Paket!!!
%%%
\makeatletter
\usepackage[printonlyused]{acronym}
\@ifpackagelater{acronym}{2015/03/20}
  {%
    \renewcommand*{\aclabelfont}[1]{\textbf{\textsf{\acsfont{#1}}}}
  }%
  {%
  }%
\makeatother
%\renewcommand*{\familydefault}{\rmdefault}
%\addtokomafont{disposition}{\rmfamily}
%
% Typewriter
%\renewcommand*{\familydefault}{\ttdefault}
%\addtokomafont{disposition}{\ttfamily}

%%
%% @stud
%%
%% Uncomment the following lines to support hard URL breaks in bibliography
%\apptocmd{\UrlBreaks}{\do\f\do\m}{}{}
%\setcounter{biburllcpenalty}{9000}% Kleinbuchstaben
%\setcounter{biburlucpenalty}{9000}% Großbuchstaben

%%
%% @stud
%%
%% FOOTNOTES: Count footnotes over chapters
%% \counterwithout{footnote}{chapter}

%	ACRONYMS
\makeatletter
\@ifpackagelater{acronym}{2015/03/20}
{\renewcommand*{\aclabelfont}[1]{\textbf{{\acsfont{#1}}}}}{}
\makeatother

%	LISTINGS
% @stud: ggf. Namen/Text anpassen (englisch)
\renewcommand{\lstlistingname}{Source}
\renewcommand{\lstlistlistingname}{Source directory}
\lstset{numbers=left,
	numberstyle=\tiny,
	captionpos=b,
	basicstyle=\ttfamily\small}

%	ALGORITHMS
% @stud: ggf. Namen/Text anpassen (englisch)
\renewcommand{\listalgorithmname}{Algorithm directory}
\floatname{algorithm}{Algorithm}

%	PAGE HEADER / FOOTER
%	Warning: There are some redefinitions throughout the master.tex-file!  DON'T CHANGE THESE REDEFINITIONS!
\RequirePackage[automark]{scrlayer-scrpage}
%alternatively with separation lines: \RequirePackage[automark,headsepline,footsepline]{scrlayer-scrpage}

\renewcommand{\chaptermarkformat}{}
\RedeclareSectionCommand[beforeskip=0pt]{chapter}
\clearpairofpagestyles

%\ifoot[\rule{0pt}{\ht\strutbox+\dp\strutbox}DHBW Mannheim]{\rule{0pt}{\ht\strutbox+\dp\strutbox}DHBW Mannheim}
\ofoot[\rule{0pt}{\ht\strutbox+\dp\strutbox}\pagemark]{\rule{0pt}{\ht\strutbox+\dp\strutbox}\pagemark}
\ohead{\headmark}

% Definitionen und Commands
\newcommand{\indextype}{numeric}
\newcommand{\abs}{\par\vskip 0.2cm\goodbreak\noindent}
\newcommand{\nl}{\par\noindent}
\newcommand{\mcl}[1]{\mathcal{#1}}
\newcommand{\nowrite}[1]{}
\newcommand{\NN}{{\mathbb N}}
\newcommand{\imagedir}{img}
\newcommand{\TitelDerArbeit}[1]{\def\DerTitelDerArbeit{#1}\hypersetup{pdftitle={#1}}}
\newcommand{\AutorDerArbeit}[1]{\def\DerAutorDerArbeit{#1}\hypersetup{pdfauthor={#1}}}
\newcommand{\Vorlesung}[1]{\def\DieVorlesung{#1}}
\newcommand{\Firma}[1]{\def\DerNameDerFirma{#1}}
\newcommand{\Kurs}[1]{\def\DieKursbezeichnung{#1}}
\newcommand{\Abteilung}[1]{\def\DerNameDerAbteilung{#1}}
\newcommand{\Studiengangsleiter}[1]{\def\DerStudiengangsleiter{#1}}
\newcommand{\WissBetreuer}[1]{\def\DerWissBetreuer{#1}}
\newcommand{\FirmenBetreuer}[1]{\def\DerFirmenBetreuer{#1}}
\newcommand{\Bearbeitungszeitraum}[1]{\def\DerBearbeitungszeitraum{#1}}
\newcommand{\Abgabedatum}[1]{\def\DasAbgabedatum{#1}}
\newcommand{\Matrikelnummer}[1]{\def\DieMatrikelnummer{#1}}
\newcommand{\Studienrichtung}[1]{\def\DieStudienrichtung{#1}}
\newcommand{\ArtDerArbeit}[1]{\def\DieArtDerArbeit{#1}}
\newcommand{\Literaturverzeichnis}{Bibliography}

\newcommand{\settingBibFootnoteCite}{
	\setlength{\bibparsep}{\parskip}		  % Add some space between biblatex entries in the bibliography
	\addbibresource{bibliography.bib}	    % Add file bibliography.bib as biblatex resource
	\DefineBibliographyStrings{ngerman}{andothers = {{et\,al\adddot}},}
}

\newcommand{\setTitlepage}{
	% !TEX root =  master.tex
% @stud: ggf. Namen/Text anpassen (englisch)
\begin{titlepage}
\vspace{1em}
%\sffamily
\begin{center}
	{\textsf{\large Mannheim Business School}}\\[4em]
	{\textsf{\textbf{\large{Assignment}}}\\[6mm]
	{\textsf{\textbf{\Large{}\DerTitelDerArbeit}}} \\[1.5cm]
	{\textsf{\textbf{\large{}Master in Sustainability and Impact Management}}\\[6mm]
	\textsf{\textbf{Module - \DieVorlesung }}}\vspace{10em}
	
	\begin{minipage}{\textwidth}
		\begin{tabbing}
		Wissenschaftliche(r) Betreuer(in): \hspace{0.85cm}\=\kill
        Author: \> \DerAutorDerArbeit \\[1.5mm]
        Instructor of the course: \> \DerWissBetreuer \\[1.5mm]
		Course: \> \DieKursbezeichnung \\[1.5mm]
		Processing period: \> \DerBearbeitungszeitraum\\[1.5mm]
%		alternativ:\\[1.5mm]
%		Eingereicht: \> \DasAbgabedatum	
		\end{tabbing}
	\end{minipage}
\end{center}
\end{titlepage}
	\pagenumbering{roman} % Römische Seitennummerierung
	\normalfont
}

\newcommand{\initializeText}{
	\clearpage
	\ihead{\chaptername~\thechapter} % Neue Header-Definition
	\pagenumbering{arabic}           % Arabische Seitenzahlen
}

\newcommand{\initializeBibliography}{
	\ihead{}
	\printbibliography[title=\Literaturverzeichnis]
	\cleardoublepage
}

\newcommand{\initializeAppendix}{
	\appendix
  \ihead{}
  \cftaddtitleline{toc}{chapter}{Appendix}{}
}
%%%%%%%%%%%%%%%%%%%%%%%%%%%%%%%%%%%
% LITERATURVERZEICHNIS
% @stud: Literaturverzeichnis in Datei bibliography.bib anpassen.
%
% [Alternative zu Verwendung von \initializeBibliography: Citavi ... (dazu eigenes LaTex Coding verwenden)]
%
\addbibresource{bibliography.bib}
%\DefineBibliographyStrings{ngerman}{andothers = {{et\,al\adddot}},}

% Elementare Konfigurationen und Definitionen werden geladen
% @stud: gegebenenfalls anpassen
%


% @stud
%
