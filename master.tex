\documentclass[fontsize=12pt,BCOR=5mm,DIV=12,parskip=half,listof=totoc,
           paper=a4,toc=bibliography=totoc,pointlessnumbers,plainfootsepline]{scrreprt}

\usepackage{fontspec}
\setmainfont{Times New Roman}
\usepackage{setspace}
\doublespacing % Enable double spacing

\usepackage[english]{babel}
\usepackage{csquotes}
%%%%%%%%%%%%%
%% ZITIERSTIL
%%%%%%%%%%%%%
%
% @stud: Zitierstil in package biblatex unten wählen
%
% NUMERIC Style - e. g. [12]
% style=numeric 
%
% IEEE Style - numeric kind of style 
% style=ieee 
%
% ALPHABETIC Style - e. g. [AB12]
% style=alphabetic 
%
% HARVARD Style 
% style=apa 
%
% CHICAGO Style 
% style=authoryear
%
% Position des Zitats:
%
% autocite=inline 
%
% (!!) FOOTNOTE POSITION NOT RECOMMENDED IN MINT DOMAIN:
% autocite=footnote
%
%
\usepackage[backend=biber, autocite=footnote, style=authoryear, dashed=false]{biblatex} 
\usepackage{makeidx}                  % allows index generation
\usepackage{listings}	                %Format Listings properly
\usepackage{lipsum}                   % Blindtext
\usepackage{graphicx}                 % use various graphics formats
%\usepackage[german]{varioref}         % nicer references \vref
\usepackage{caption}	                % better Captions
\usepackage{booktabs}                 % nicer Tabs
\usepackage[hidelinks=true]{hyperref} % keine roten Markierungen bei Links
\usepackage{fnpct}                    % Correct superscripts 
\usepackage{calc}                     % Used for extra space below footsepline, in particular
\usepackage{array}
\usepackage{acronym}
\usepackage{algorithm}
\usepackage{algpseudocode}
\usepackage{setspace}
\usepackage{tocloft}
%\usepackage[T1]{fontenc}

% Definitionen und Commands
\newcommand{\indextype}{numeric}
\newcommand{\abs}{\par\vskip 0.2cm\goodbreak\noindent}
\newcommand{\nl}{\par\noindent}
\newcommand{\mcl}[1]{\mathcal{#1}}
\newcommand{\nowrite}[1]{}
\newcommand{\NN}{{\mathbb N}}
\newcommand{\imagedir}{img}
\newcommand{\TitelDerArbeit}[1]{\def\DerTitelDerArbeit{#1}\hypersetup{pdftitle={#1}}}
\newcommand{\AutorDerArbeit}[1]{\def\DerAutorDerArbeit{#1}\hypersetup{pdfauthor={#1}}}
\newcommand{\Vorlesung}[1]{\def\DieVorlesung{#1}}
\newcommand{\Firma}[1]{\def\DerNameDerFirma{#1}}
\newcommand{\Kurs}[1]{\def\DieKursbezeichnung{#1}}
\newcommand{\Abteilung}[1]{\def\DerNameDerAbteilung{#1}}
\newcommand{\Studiengangsleiter}[1]{\def\DerStudiengangsleiter{#1}}
\newcommand{\WissBetreuer}[1]{\def\DerWissBetreuer{#1}}
\newcommand{\FirmenBetreuer}[1]{\def\DerFirmenBetreuer{#1}}
\newcommand{\Bearbeitungszeitraum}[1]{\def\DerBearbeitungszeitraum{#1}}
\newcommand{\Abgabedatum}[1]{\def\DasAbgabedatum{#1}}
\newcommand{\Matrikelnummer}[1]{\def\DieMatrikelnummer{#1}}
\newcommand{\Studienrichtung}[1]{\def\DieStudienrichtung{#1}}
\newcommand{\ArtDerArbeit}[1]{\def\DieArtDerArbeit{#1}}
\newcommand{\Literaturverzeichnis}{Literaturverzeichnis}

% Page Layout
\oddsidemargin=0mm
\evensidemargin=0mm
\textwidth=159mm
\topmargin=-18mm
\headsep=10mm
\textheight=251mm
\footheight=15mm

\makeindex

%%%%%%%%%%%%%%%%%%%%%%%%%%%%%%%%%%%
% LITERATURVERZEICHNIS
% @stud: Literaturverzeichnis in Datei bibliography.bib anpassen. 
%
% [Alternative zu Verwendung von \initializeBibliography: Citavi ... (dazu eigenes LaTex Coding verwenden)]
%
\addbibresource{bibliography.bib}
%\DefineBibliographyStrings{ngerman}{andothers = {{et\,al\adddot}},}

% Elementare Konfigurationen und Definitionen werden geladen 
% @stud: gegebenenfalls anpassen
%
% !TEX root = master.tex

%%%%%%%%%%%%%%%%%%%%%%%%%%%%%%%%%%%%%%%%%%%%%%%%%%%%%%%%%%%%%%%%%%
% LANGUAGE SETTINGS
\usepackage[english]{babel}
\usepackage{csquotes}

% ESSENTIAL PACKAGES
\usepackage{makeidx}                  % allows index generation
\usepackage{listings}                 % Format Listings properly
\usepackage{lipsum}                   % Blindtext
\usepackage{graphicx}                 % use various graphics formats
\usepackage{varioref}                 % nicer references \vref
\usepackage{caption}                  % better Captions
\usepackage{booktabs}                 % nicer Tabs
\usepackage[hidelinks]{hyperref}      % no red markings on links
\usepackage{fnpct}                    % Correct superscripts
\usepackage{calc}                     % Used for extra space below footsepline, in particular
\usepackage{array}
\usepackage{acronym}
\usepackage{algorithm}
\usepackage{algpseudocode}
\usepackage{setspace}
\usepackage{tocloft}

% FONT SELECTION
\usepackage{fontspec}
\setmainfont{Times New Roman}
\doublespacing

% Adjust default family to serif (Times New Roman)
\renewcommand*{\familydefault}{\rmdefault}
\addtokomafont{disposition}{\rmfamily}

\usepackage{pgfplots}
\pgfplotsset{compat=newest}

\usepackage[a4paper, margin=1in]{geometry}
\pgfplotsset{width=14cm,height=10cm,compat=1.9}
\usepackage{array}
\newcolumntype{L}[1]{>{\raggedright\arraybackslash}p{#1}}
\newcolumntype{C}[1]{>{\centering\arraybackslash}p{#1}}

\usepackage{tikz}
\usetikzlibrary{positioning, arrows.meta}

% BIBLIOGRAPHY SETTINGS
\usepackage[backend=biber, autocite=inline, style=authoryear]{biblatex} % Set the style here
\addbibresource{bibliography.bib}

% FOOTNOTES
\usepackage{chngcntr}
\counterwithout{footnote}{chapter}

% ACRONYMS
\makeatletter
\usepackage[printonlyused]{acronym}
\@ifpackagelater{acronym}{2015/03/20}
  {%
    \renewcommand*{\aclabelfont}[1]{\textbf{\textsf{\acsfont{#1}}}}
  }%
  {%
  }%
\makeatother

% LISTINGS
\renewcommand{\lstlistingname}{Source}
\renewcommand{\lstlistlistingname}{Source directory}
\lstset{numbers=left,
	numberstyle=\tiny,
	captionpos=b,
	basicstyle=\ttfamily\small}

% ALGORITHMS
\renewcommand{\listalgorithmname}{Algorithm directory}
\floatname{algorithm}{Algorithm}

% PAGE HEADER / FOOTER
\RequirePackage[automark]{scrlayer-scrpage}
% alternatively with separation lines: \RequirePackage[automark,headsepline,footsepline]{scrlayer-scrpage}

\renewcommand{\chaptermarkformat}{}
\RedeclareSectionCommand[beforeskip=0pt]{chapter}
\clearpairofpagestyles

\ofoot[\rule{0pt}{\ht\strutbox+\dp\strutbox}\pagemark]{\rule{0pt}{\ht\strutbox+\dp\strutbox}\pagemark}
\ohead{\headmark}

% Definitionen und Commands
\newcommand{\indextype}{numeric}
\newcommand{\abs}{\par\vskip 0.2cm\goodbreak\noindent}
\newcommand{\nl}{\par\noindent}
\newcommand{\mcl}[1]{\mathcal{#1}}
\newcommand{\nowrite}[1]{}
\newcommand{\NN}{{\mathbb N}}
\newcommand{\imagedir}{img}
\newcommand{\TitelDerArbeit}[1]{\def\DerTitelDerArbeit{#1}\hypersetup{pdftitle={#1}}}
\newcommand{\AutorDerArbeit}[1]{\def\DerAutorDerArbeit{#1}\hypersetup{pdfauthor={#1}}}
\newcommand{\Vorlesung}[1]{\def\DieVorlesung{#1}}
\newcommand{\Firma}[1]{\def\DerNameDerFirma{#1}}
\newcommand{\Kurs}[1]{\def\DieKursbezeichnung{#1}}
\newcommand{\Abteilung}[1]{\def\DerNameDerAbteilung{#1}}
\newcommand{\Studiengangsleiter}[1]{\def\DerStudiengangsleiter{#1}}
\newcommand{\WissBetreuer}[1]{\def\DerWissBetreuer{#1}}
\newcommand{\FirmenBetreuer}[1]{\def\DerFirmenBetreuer{#1}}
\newcommand{\Bearbeitungszeitraum}[1]{\def\DerBearbeitungszeitraum{#1}}
\newcommand{\Abgabedatum}[1]{\def\DasAbgabedatum{#1}}
\newcommand{\Matrikelnummer}[1]{\def\DieMatrikelnummer{#1}}
\newcommand{\Studienrichtung}[1]{\def\DieStudienrichtung{#1}}
\newcommand{\ArtDerArbeit}[1]{\def\DieArtDerArbeit{#1}}
\newcommand{\Literaturverzeichnis}{Bibliography}

\newcommand{\settingBibFootnoteCite}{
	\setlength{\bibparsep}{\parskip}		  % Add some space between biblatex entries in the bibliography
	\addbibresource{bibliography.bib}	    % Add file bibliography.bib as biblatex resource
	\DefineBibliographyStrings{ngerman}{andothers = {{et\,al\adddot}},}
}

\newcommand{\setTitlepage}{
	% !TEX root =  master.tex
% @stud: ggf. Namen/Text anpassen (englisch)
\begin{titlepage}
\vspace{1em}
%\sffamily
\begin{center}
	{\textsf{\large Mannheim Business School}}\\[4em]
	{\textsf{\textbf{\large{Assignment}}}\\[6mm]
	{\textsf{\textbf{\Large{}\DerTitelDerArbeit}}} \\[1.5cm]
	{\textsf{\textbf{\large{}Master in Sustainability and Impact Management}}\\[6mm]
	\textsf{\textbf{Module - \DieVorlesung }}}\vspace{10em}
	
	\begin{minipage}{\textwidth}
		\begin{tabbing}
		Wissenschaftliche(r) Betreuer(in): \hspace{0.85cm}\=\kill
        Author: \> \DerAutorDerArbeit \\[1.5mm]
        Instructor of the course: \> \DerWissBetreuer \\[1.5mm]
		Course: \> \DieKursbezeichnung \\[1.5mm]
		Processing period: \> \DerBearbeitungszeitraum\\[1.5mm]
%		alternativ:\\[1.5mm]
%		Eingereicht: \> \DasAbgabedatum	
		\end{tabbing}
	\end{minipage}
\end{center}
\end{titlepage}
	\pagenumbering{roman} % Römische Seitennummerierung
	\normalfont
}

\newcommand{\initializeText}{
	\clearpage
	\ihead{\chaptername~\thechapter} % Neue Header-Definition
	\pagenumbering{arabic}           % Arabische Seitenzahlen
}

\newcommand{\initializeBibliography}{
	\ihead{}
	\printbibliography[title=\Literaturverzeichnis]
	\cleardoublepage
}

\newcommand{\initializeAppendix}{
	\appendix


% @stud
%
% PERSÖNLICHE ANGABEN (BITTE VOLLSTÄNDIG EINGEBEN zwischen den Klammern: {...})
%
\TitelDerArbeit{The practical implementation of the EU taxonomy at SAP}
\AutorDerArbeit{Michael Spilger}
\Kurs{MSIM25}
\Vorlesung{Non-Financial Performance Evaluation \& Reporting}
\Bearbeitungszeitraum{19.04.2024-17.05.2024}
\WissBetreuer{Dr. Stefanie Ahrens}

\begin{document}

\setTitlepage

%%%%%%%%%%%%%%%%%%%%%%%%%%%%%%%%%%%
% EHRENWÖRTLICHE ERKLÄRUNG
%
% @stud: ewerkl.tex bearbeiten
%
% !TEX root =  master.tex
\clearpage
\chapter*{Declaration of Authorship}

% Wird die folgende Zeile auskommentiert, erscheint die ehrenwörtliche
% Erklärung im Inhaltsverzeichnis.

% \addcontentsline{toc}{chapter}{Ehrenwörtliche Erklärung}
\begin{center}
(Course: \textit{\DieKursbezeichnung})

"I hereby declare that the paper presented ``\textit{\DerTitelDerArbeit}'' is my own work and that I have not called upon the help of a third party. In addition, I affirm that neither I nor anybody else has submitted this paper or parts of it to obtain credits elsewhere before. I have clearly marked and acknowledged all quotations or references that have been taken from the works of others. All secondary literature and other sources are marked and listed in the bibliography. The same applies to all charts, diagrams and illustrations as well as to all Internet resources. Moreover, I consent to my paper being electronically stored and sent anonymously in order to be checked for plagiarism."

\vspace{3cm}
\DerAutorDerArbeit

Mannheim, \today 

\end{center} 
\cleardoublepage  
%%%%%%%%%%%%%%%%%%%%%%%%%%%%%%%%%%%

%%%%%%%%%%%%%%%%%%%%%%%%%%%%%%%%%%%
% SPERRVERMERK
%
% @stud: nondisclosurenotice.tex bearbeiten
%
%% !TEX root =  master.tex
\chapter*{Sperrvermerk}

\begin{center}
\fbox{
		\begin{minipage}{33em}
			\textbf{Ein Sperrvermerk sollte nur bei berechtigtem Bedarf gesetzt werden!\\[10pt] 
				Beachten Sie, dass mit Sperrvermerk	versehene Arbeiten nicht für weitere wissenschaftliche Zwecke 
				außerhalb des Firmenkontextes oder zur Publikation verwendet werden dürfen.\\[10pt]
				Wir empfehlen, wenn m\"oglich, auf den Sperrvermerk zu verzichten.\\[10pt]
				Besprechen Sie diese Problematik mit Ihrer Firma!}
		\end{minipage}
}
\end{center}

(Mustertext) Der Inhalt dieser Arbeit darf weder als Ganzes noch in Auszügen Personen außerhalb des Prüfungsprozesses 
und des Evaluationsverfahrens zugänglich gemacht werden, sofern keine anders lautende Genehmigung der Ausbildungsstätte vorliegt. 

\cleardoublepage
 
%\cleardoublepage
%%%%%%%%%%%%%%%%%%%%%%%%%%%%%%%%%%%

%%%%%%%%%%%%%%%%%%%%%%%%%%%%%%%%%%%
%	KURZFASSUNG
%
% @stud: acknowledge.tex bearbeiten
%
%% !TEX root =  master.tex
\chapter*{Danksagung}

Hier können Sie eine Danksagung schreiben. 



%\cleardoublepage 
%%%%%%%%%%%%%%%%%%%%%%%%%%%%%%%%%%%

%%%%%%%%%%%%%%%%%%%%%%%%%%%%%%%%%%%
% VERZEICHNISSE und ABSTRACT
%
% @stud: ggf. nicht benötigte Verzeichnisse auskommentieren/löschen
%
\tableofcontents
\cleardoublepage

% Abbildungsverzeichnis
%\phantomsection
%\addcontentsline{toc}{chapter}{\listfigurename}
%\listoffigures
%\cleardoublepage

%	Tabellenverzeichnis
\phantomsection
%\addcontentsline{toc}{chapter}{\listtablename}
%\listoftables
%\cleardoublepage

%	Listingsverzeichnis / Quelltextverzeichnis
%\lstlistoflistings
%\cleardoublepage

% Algorithmenverzeichnis
%\listofalgorithms
%\cleardoublepage

% Abkürzungsverzeichnis
% @stud: acronyms.tex bearbeiten
% !TEX root =  master.tex
\clearpage
\chapter*{List of abbreviations}	
\addcontentsline{toc}{chapter}{List of abbreviations}

\begin{acronym}
    \acro{SAP}{System Analysis Program Development}
    \acro{DNSH}{Does not significantly harm}
    \acro{CAPEX}{Capital Expenditures}
    \acro{OPEX}{Operational Expenditures}
\end{acronym} 
\cleardoublepage

\onehalfspacing

%	Kurzfassung / Abstract
% @stud: abstract.tex bearbeiten
%% !TEX root =  master.tex
%%
%% @stud: sprachspezifische Anpassung
%%
\chapter*{Abstract}
\addcontentsline{toc}{chapter}{Abstract}

\begin{minipage}{\textwidth}
		\begin{tabbing}
		Wissenschaftliche(r) Betreuer(in): \hspace{0.85cm}\=\kill
		Author: \> \DerAutorDerArbeit \\[1.5mm]
		Title: \> \DerTitelDerArbeit \\[1.5mm]
%		alternativ:\\[1.5mm]
%		Eingereicht: \> \DasAbgabedatum	
		\end{tabbing}
	\end{minipage} 
\cleardoublepage

\initializeText

%%%%%%%%%%%%%%%%%%%%%%%%%%%%%%%%%%%%%%%%%%%%%%%%%%%%%%%%%%%%%%%%%%%%%%%%%%%%%%%%%%%%%%%%%%
% KAPITEL UND ANHÄNGE
%
% @stud:
%   - nicht benötigte: auskommentieren/löschen
%   - neue: bei Bedarf hinzufügen mittels input-Kommando an entsprechender Stelle einfügen
%%%%%%%%%%%%%%%%%%%%%%%%%%%%%%%%%%%%%%%%%%%%%%%%%%%%%%%%%%%%%%%%%%%%%%%%%%%%%%%%%%%%%%%%%%

%%%%%%%%%%%%%%%%%%%%%%%%%%%%%%%%%%%
% KAPITEL
%
% @stud: einzelne Kapitel bearbeiten und eigene Kapitel hier einfügen
%
% Einleitung
% !TEX root =  master.tex
\chapter{Introduction}

%\autocites[Cf.][1-4]{tax1}[Cf.]{eu_tax_sus}

% mehrere Grundlagen- und Forschungs-Kapitel
% !TEX root =  master.tex
\chapter{Main EU taxonomy requirements for NFEs}
Article 3 of the EU Taxonomy sets out four criteria for environmentally sustainable economic activities.

The first criterion in Article 3(a) states that an economic activity must make a significant contribution to one or more of the environmental objectives set out in Articles 10 to 16. These objectives include climate change mitigation and adaptation, the sustainable use and protection of water and marine resources, the transition to a circular economy, pollution prevention and control, and the protection and restoration of biodiversity and ecosystems. This contribution is assessed on the basis of specific indicators and thresholds set out in the technical screening criteria to ensure that the activity is in line with the overall objectives of sustainability and environmental protection.
\autocites[Cf.][Articles 3, 10-16]{eu_regulation}

The second criterion is that the activity must not significantly harm another environmental objective referred to in Article 9, which is compatible with Article 17. This principle is known as \ac{DNSH}. According to Article 17, an activity is considered to significantly harm environmental objectives if it leads to significant adverse effects on any of the following objectives: Climate change mitigation, adaptation to climate change, sustainable use and protection of water and marine resources, transition to a circular economy, pollution prevention and control, and protection and restoration of biodiversity and ecosystems.
\autocites[Cf.][Articles 3, 9, 17]{eu_regulation}


The third criterion is that the activity is carried out in compliance with the minimum guarantees defined in Article 18. These minimum safeguards ensure that the activity complies with high social and corporate governance standards and, in particular, with guidelines such as the OECD Guidelines for Multinational Enterprises and the UN Guiding Principles on Business and Human Rights. This compliance is crucial in order to maintain the integrity and ethical standards of the activity and at the same time contribute to environmental sustainability.\autocites[Cf.][Article 3, 18]{eu_regulation}


The fourth criterion states that a regulation must fulfil the technical screening criteria of Articles 10(3), 11(3), 12(2), 13(2), 14(2) or 15(2). These technical screening criteria contain detailed benchmarks and thresholds that an activity must fulfil in order to be considered as contributing substantially to one or more of the environmental objectives, such as climate change mitigation, climate change adaptation, sustainable use and protection of water and marine resources, transition to a circular economy, pollution prevention and control, and protection and restoration of biodiversity and ecosystems. In particular the following\autocites[Cf.][Article 3, 12-15]{eu_regulation}:
\begin{itemize}
    \item \textbf{Article 10 (3)}: Climate change mitigation, such as improving energy efficiency or producing renewable energy.
    \item \textbf{Article 11 (3)}: Climate change adaptation, including measures that reduce adverse climate impacts.
    \item \textbf{Article 12 (2)}: Sustainable use and protection of water and marine resources, enhancing water quality.
    \item \textbf{Article 13 (2)}: Transition to a circular economy, emphasizing resource efficiency and waste reduction.
    \item \textbf{Article 14 (2)}: Pollution prevention and control, minimizing emissions and waste.
    \item \textbf{Article 15 (2)}: Protection and restoration of biodiversity and ecosystems, supporting natural habitats.
\end{itemize}


According to Article 3 of the EU Taxonomy, the assessment of the environmental sustainability of an economic activity includes three steps: Firstly, specific indicators and thresholds set out in the technical screening criteria are used to assess whether the activity contributes significantly to one or more environmental objectives. Secondly, similar indicators and thresholds are used to ensure that the activity does not significantly harm any of the other environmental objectives in accordance with the \ac{DNSH} principle. Thirdly, confirmation that the activity complies with the minimum protection requirements, whereby the fulfilment of all these criteria results in the activity being classified as "taxonomy-compliant". 
\autocites[Cf.]{eu_regulation}[Cf.][10]{tax1}


% !TEX root =  master.tex
\chapter{SAPs presentation of the reporting requirements}
This chapter provides an analysis of how SAP, a leading multinational software corporation, aligns its reporting with the EU Taxonomy Regulation. It will be explored the extent of SAP's business coverage under the EU Taxonomy, the reporting of taxonomy-aligned revenue and capital expenditures (CAPEX), and evaluated the suitability of the taxonomy as a tool for assessing the company's sustainability efforts.
\section{Coverage of business by EU taxonomy}
SAP identifies two taxonomy eligable activities in their report. The activities are 
\begin{itemize}
    \item Activity 8.1 “Data processing, hosting, and related activities” (substantial contribution to climate change mitigation) - SAP has assessed its data centres and cloud solutions, but has concluded that they do not meet the criteria for an adjustment to the taxonomy in 2023. The problems include the refrigerants, which have too high a potential for global warming, but whose immediate replacement would generate toxic waste, so it is not helpful to replace them immediately. In addition, the data centres, colocations and hyperscalers do not meet all the criteria. SAP is committed to achieving harmonisation within the next few years.
    \item Activity 3.3 “Manufacture of low carbon technologies for transport” (substantial contribution to climate change mitigation) - Here the SAP car fleet is controbuting to climate change mitigation. In 2023 SAP capitalized €210 million in this activity and reports €51 million as taxonomy aligned. SAP is commited to buy more taxonomy aligned vehicles in the future.
    \item Other Activities - Combined Heat/Cool and Power Generation Facilities which process fossil gaeous fuels are operated by SAP in single locations. SAP reports that the amount is negligible.
\end{itemize}
In numbers, this means that SAP has 43\% of taxonomy-eligible revenues, 19\% of taxonomy-eligible \ac{OPEX} and 27\% of taxonomy-eligible activities in \ac{CAPEX}.  In each category in which the business is measured according to the EU taxonomy, SAP is only represented with less than half to less than a quarter of the total business.
\autocites[Cf.][92-98]{sapreport}


\section{Reporting of taxonomy aligned revenue and CAPEX}
SAP reports taxonomy-aligned revenue and \ac{CAPEX} separately. SAP has 43\% taxonomy-eligible activities for \textbf{revenue}. None of them are taxonomy-aligned. This makes €13.361 million taxonomy-eligible but unaligned revenue. 57\% (€17.846 million) are not eligible for the EU taxonomy. The situation is different with the \textbf{\ac{CAPEX}}. Here, 27\% (€370 million) is taxonomy-eligible and 73\% (€1,008 million) is not taxonomy-eligible. 4\% of \ac{CAPEX} are taxonomy-eligible and aligned. This results from the above-mentioned activity 3.3, which relates to the electric cars in SAP's fleet.
\autocites[Cf.][92-98]{sapreport}

\section{Suitability of the taxonomy to judge the companies sustainability}
The analysis of the business areas covered by SAP under the EU Taxonomy shows that while the company has identified and reported on taxonomy-eligible activities, progress is still needed to bring these activities into taxonomy-compliant status. In the areas of data processing and fleet management in particular, considerable efforts are being made to fulfil the technical screening criteria and align with the environmental objectives set out in the EU taxonomy. If this is done based on the EU taxonomy, that would be good, but SAP already reports to be carbon neutral. In terms of resource consumption, waste and $CO_2e$ emissions, SAP has not only significantly reduced its environmental impact, but also claims to be climate neutral. This shows that the taxonomy is not suitable for assessing SAP's sustainability as a complete company, as SAP's business activities are only covered to a small extent, but it also shows that there are many technical details to think about when thinking about sustainability. SAP may not have focussed sufficiently on technical criteria, but merely sought to report carbon neutrality, even if that means not replacing refrigerants with those that produce less or no toxic waste. Therefore, technical criteria are needed to assess SAP's sustainability and it is good if business activities are included, but the percentage of business units included should be higher to make companies as a whole comparable and assessable.
\autocites[Cf.][110-115]{sapreport}[Cf.][276-278]{sapreport}
% Fazit und Ausblick
%% !TEX root =  master.tex
\chapter{Conclusion}

%%%%%%%%%%%%%%%%%%%%%%%%%%%%%%%%%%%

%%%%%%%%%%%%%%%%%%%%%%%%%%%%%%%%%%%
% ANHÄNGE
%
% @stud: einzelne Anhänge bearbeiten und eigene Anhänge hier einfügen 
%        die nachfolgenden Zeilen deaktivieren, wenn keine Anhänge verwendet werden
%
%\initializeAppendix
%% !TEX root =  master.tex

%% !TEX root =  master.tex
\chapter{Beispiel-Anhang: Noch ein Testanhang}
nochmal: lipsum ...

%%%%%%%%%%%%%%%%%%%%%%%%%%%%%%%%%%%

\singlespacing

%\ihead{}
%\printbibliography[title=\Literaturverzeichnis] 
\printbibliography 
\cleardoublepage

%\initializeBibliography
%%%%%%%%%%%%%%%%%%%%%%%%%%%%%%%%%%%

%%%%%%%%%%%%%%%%%%%%%%%%%%%%%%%%%%%
% INDEX
% @stud: ggf. Index auskommentieren, wenn nicht benötigt
%
%\addcontentsline{toc}{chapter}{Index}
%\printindex

\end{document}
